\section{Mengen}

\subsection{Grundbegriffe}
\begin{defi}
Eine Menge ist eine Zusammenfassung von bestimmten, wohlunterscheidbaren Objekten zu einem Ganzen.
\end{defi}

\begin{bsp}
\begin{eqnarray*} 
A&=\left\{a,b,c,\dots ,x,y,z\right\} \textit{ Menge der kleinen lateinischen Buchstaben.} \\
\mathbb{N}&=\left\{1,2,3,\dots\right\} \textit{ Menge der natürlichen Zahlen.} \\
\\
\textit{In diesem Beispiel werden Mengen durch Aufzählung ihrer Objekte in }\left\{ \right\}\textit{ beschrieben.} \\
\textit{Man kann sie auch durch Eigenschaften beschreiben: }\\
W=\left\{w|\textit{w hat eine Eigenschaft}\right\}
\end{eqnarray*}
\end{bsp}

